The goal of this research is to divert I/O resources to where they are needed most. 
%
Our technique consists of two phases. 
%
In the first phase, we perform a lightweight analysis step to determine the most pertinent data among all the compute nodes. 
%
In the second phase, we perform wavelet compression, adapting the I/O budget per node according to results from the first phase.

Within the first phase of our technique, we explore two different strategies for assessing data content.
%
Our first strategy is inherent to wavelet compression --- utilizing the coefficient magnitudes that are calculated during the transform process. 
%
Our second strategy uses Shannon entropy to discern the most salient data;
these results are then used to derive the budgets for each compute node's wavelet compression.
%
We also compare these two strategies against the standard strategy, i.e.,
equally allocating the I/O budget to all compute nodes. 
%These strategies are compared against the standard model for wavelet compression, which 

\vspace{-1.3em}
\subsection{Wavelet Magnitude}
%
The first strategy, involving the calculated wavelet magnitudes, is innate to wavelet compression. 
%
We note that in the setting of multi-domain simulations (domain meaning the
data set owned by one rank), 
compression is applied on individual domains to generate coefficients.
%
Magnitude difference across domains no longer necessarily warrant difference in information content; 
nevertheless, magnitude still remains as a strong indicator of the global importance of each domain.
%
%During the wavelet compression process, the data values are prioritized with each pass of the filter along each axis. 
%
With this magnitude-based strategy, first, each rank will calculate its total local magnitude.
%
Second, an \texttt{MPI$\_$AllReduce} summation will calculate the total global magnitude.
%
And then lastly, each rank will determine their respective ratio. 
%
This ratio is then used to determine that rank's I/O budget. 

\vspace{-1.3em}
\subsection{Entropy}
The second strategy utilizes Shannon Entropy, also called Information Entropy, a widely used strategy to determine the importance of data. 
%
From a high level, Shannon Entropy calculated the number of bits required to save the given data. 
%
The more bits that are required, then the more information that is present. 
%
%Entropy is calculated by binning the data into some $n$ bins. 
%
%Once all the data has been binned, bin $i$ will calculate its respective probability $p_i$.
%
%Let $b_i$ be the total number of values in bin $i$. 
%
%Then
%\begin{center}
% $p_i = \frac{b_i}{\text{total number of values}}$.
%\end{center} 
%
%From there, entropy
%\begin{center}
% $E = - \sum_{i=1}^n p_i \log p_i$. 
%\end{center}
%

Each rank will calculate its local entropy value from the input data. 
%
Then, an \texttt{MPI$\_$AllReduce} summation is used to calculate the total global entropy, and from there each rank will calculate their respective ratio. 
%
As with the wavelet strategy, the calculated ratio will determine each ranks I/O budget and will save data accordingly.  

\vspace{-1.3em}
\subsection{Standard}
The two reallocation strategies will be compared against the standard wavelet compression where each rank has the same I/O budget. 
