Visualization under the old paradigm required data producers to save out every cycle, at which point the data would be visualized post-hoc. 
%
But the increasing gap between compute and I/O capabilities has made the old post-hoc paradigm unpalatable. 
%
This has lead to new approaches, namely in situ visualization and analysis. 
%
The former produces images as the simulation runs, but no data is saved. 
%
The latter performs lightweight analysis that should reduce the amount of data being saved, allowing for post-hoc analysis. 
%
This paper focuses on the latter technique. 
%
And more specifically, this research focuses on in situ compression. 


For large-scale simulations, saving every cycle is a non-starter. 
%
Instead, simulations will save individual cycles at fixed or variable intervals. 
%
And even then, those chosen cycles will utilize some form of compression to further reduce the I/O burden. 
%
One such compression technique is wavelet compression. 
%
In a typical workflow, wavelet compression concentrates the vast majority of 
information into a small amount of coefficients, 
and given an I/O budget, it fills the budget with coefficients with the most 
information content.

Typically, during a large-scale, parallel simulation, each domain is allocated the same amount of resources, especially when it comes to the desired I/O budget.
%
In practice, however, some domains may contain data of little consequence, and thus, their resources may be better utilized elsewhere. 
%
This work researches the effectiveness of resource reallocation for compression with two strategies to reallocate the I/O budget of each domain for each cycle as the simulation runs. 

The first reallocation strategy is specific to wavelet compression. 
%
Wavelet compression inherently prioritizes data within a domain, we use this information to calculate a domain's data's global importance, and reallocate the I/O allocation accordingly. 
%
The second reallocation strategy incorporates Shannon Entropy, a calculation that has become common in information science and determining the information content of data. 
%
The Shannon Entropy is calculated for each domain, and then compared globally; a domain's I/O budget is then determined by the global entropy calculation.

This research shows that dynamically reallocating I/O budget can lead to increased storage savings and more accurate output.  
