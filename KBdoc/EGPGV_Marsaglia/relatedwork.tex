Related work is broken into three subsections. 
%
The first subsection reviews work that reallocates resources to achieve better efficiency.
%
The second subsection covers work that uses entropy in scientific visualization. 
%
The third subsection briefs how wavelet compression works and
reviews its use in scientific simulation and visualization.

\vspace{-1.3em}
\subsection{Resource Reallocation}
%Understanding how to dynamically adapt resource usage on the fly is an important consideration when trying to find the right allocation strategy.
%
Understanding workflow execution and scheduling 
%hardware 
resources accordingly is an important consideration when trying to find the right resource allocation strategy.
%
%Additionally, available resources for workflow tasks may be limited and are oftentimes variable.
%
If one task is given insufficient resources, it could become a bottleneck for the entire workflow.
%
Thus, adapting resource usage dynamically, and reallocating resources according to runtime needs will speed up the overall workflow~\cite{semreport}.
%
Labasan et~al.~\cite{Labasan:EGPGV17} studied the adaptation of different resource allocations for different visualization routines in a power-constrained environment.
%
They showed that adapting power resources on a per-node need basis performed better than a uniform resource distribution strategy.
%Their work revolves around the idea that uniform allocation of a resource across nodes is not optimal in the case where the runtime behaviors are highly variable among the nodes.
%
%To alleviate this problem, reallocating resources as necessary becomes a viable alternative that improves the overall performance of the application.

Dynamically adapting I/O resources and options according to user specifications has been shown to be a useful technique for visualization tools.
%
Dorier et~al.~\cite{Damaris} presented Damaris/Viz, an in situ visualization framework to support I/O middleware that adapts to the specific needs of simulations by using a dynamic plugin-loading architecture.
% that can support some data transformations.
%adapts to the specific needs of simulations by gathering the capabilities of several visualization tools to offer a unified data management interface.
%
ADIOS~\cite{adios}, an in situ data transformation framework, also provides a level of adaptability by providing different I/O options, thereby transparently changing how the data is processed.
%
\fix{Reminder: Sam will look into that Kenny Gruchalla paper.}

\vspace{-1.3em}
\subsection{Entropy}

%Scientific Visualization helps scientists analyze and understand data by providing a visual representation of how the data evolves throughout a simulation of some event.
%
%Because the visual analysis and the discovery process is often operated by the user through trial and error in an ad hoc manner, some important features of the data may be overlooked or left undiscovered. 
%
%Moreover, critical parameters for visualization algorithms (i.e. camera positions and directions, levels of details, isocontours, etc.) often need to be fine-tuned and frequently updated in order to obtain reliable visualization results. 
%
%One major cause of the difficulties in visual analysis of large datasets is the lack of quantitative metrics to measure the visualization quality relative to the amount of information contained in the data. 
%
%As the amount of data needing to be processed continues to grow, determining what data is critical for understanding the behavior of the simulation becomes a priority for research scientists. 
%
%While there are different techniques for determining which part of the data is important, 
Information theory has been used as a viable way to distinguish significance in parts of the data~\cite{entropyWang}. 
%
%Information theory is most helpful when researchers do not know what exact questions to ask about their data or want to find more meaningful information from that data. 
%
%Exhaustively searching through all the data for meaningful insight is inefficient and oftentimes not plausible. 
%
Using information theory concepts can help scientists determine which portions of the data contain important features~\cite{purchase}.
%
Entropy is one such information theory technique.
% that helps researchers mathematically identify and extract hidden or difficult to detect features that may be present in their data. 
%
According to Shannon~\cite{Shannon}, entropy can be calculated for a variable, $x$, to measure the average uncertainty in it. 
%
%Below is the Shannon Entropy formula \cite{entropyWang} \cite{Chen}:
%\begin{equation}
%H\big(X\big) = - \sum_{x\in\chi} p\big(x\big) \log_{2} p\big(x\big)
%\end{equation}
%Let $X$ be a discrete random variable with alphabet $\chi$ and probability mass function $p\big(x\big)$ , $x\in\chi$. 
%
%
%More that can describe the function? 
The higher the entropy, the more information content this variable has.
%
Entropy is always non-negative~\cite{CoverBook} and describes the 
%number of bits 
unit of information on average required to describe this variable~\cite{LiPaper}. 
%
%When all random variables have equal probability $\Big(p\big(x\big) = \frac{1}{n}\Big)$, $H$ is maximized. 
%
%Conversely, when the probability of a single variable is one and the rest are zeros, $H$ is minimized. 
%
%Thus, the entropy calculation determines a theoretical lower bound on the number of bits required to represent the variables without information loss \cite{LiPaper}. 

Dorier et~al.~\cite{Dorier} compare a number of different approaches, including entropy, to determine the saliency of the data, and then uses these metrics for load balancing in a time-constrained environment. 
%how that not all generated data is relevant to understanding the physical phenomena being simulated and use information theory to highlight potentially interesting regions of the data.
%
%To highlight the potentially interesting regions of data that carry important information, they use metrics based on information theory.
%
%Their proposed method redistributes blocks of data across processes in order to achieve better load balance.
%
This strategy is particularly useful when visualization resources are limited and should be allocated to the most important data. 
%
Additionally, Wang et~al.~\cite{entropyWang2, wang} demonstrate how information theory, and entropy in particular, can highlight  certain areas of time-varying data that are particularly important, but their work has not been evaluated for in situ capabilities.
%Similarly, we can also describe the conditional entropy of two variables, $H\big(X|Y\big)$ . 
%
%Conditional entropy is the entropy of a random variable conditional on the knowledge of another random variable. 
%
%The conditional entropy of different variables in a field relative to other random variables is another useful concept in scientific visualization \cite{CoverBook}. 
%
%To show this, they partitioned a large data set into blocks and studied the relative entropy of one block of data compared to a different block.
%of data, indicating if one block of data has more important data than another block.
%
%In the context of a distributed memory system, this technique can be used to determine the entropy of a data partition belonging to one process compared to another data partition. 
%
%Analyzing the relative entropy of different data partitions can indicate if one block of data has more important data than another block or not. 
%
%
%While these works demonstrate the utility of using entropy to guide visualization, our work compares...

\vspace{-1.3em}
\subsection{Wavelet Compression}
%
Wavelet compression is a class of transform-based techniques with its core operation
being wavelet transforms.
%
Such a transform decomposes data into coefficients in the wavelet domain, 
representing information in various time and frequency scales.
%
%With careful choice of wavelet kernels and transformation methods, 
%the same number of wavelet coefficients are generated to preserve all the 
%information content of the input data.
%
In the use case of compression, wavelet kernels and transformation methods 
are chosen such that the transform is 
1) \textit{nonexpansive}, meaning that the number of output coefficients equals the 
number of input data points;
2) \textit{invertible}, meaning that input data could be reconstructed in its full fidelity; 
%
%Compression using wavelets happens when a portion, but not all, of wavelet coefficients are 
%used to reconstruct the original data.
%
%Errors are introduced with the discarding of coefficients.
%
and 3) excellent in 
\textit{information concentration}, meaning that the vast majority of information in the data is 
disproportionately kept in a small amount of coefficients, namely the ones with the largest magnitudes.
%
Compression using wavelets happens when only those large-magnitude coefficients are 
used to reconstruct the data, and the rest coefficients are discarded.
%
%The Haar and CDF wavelet kernels~\cite{cohen1992biorthogonal}
%exhibit all the properties mentioned above, thus are most often used for compression.


%With this study, we look into the information content of wavelet coefficients from 
%different simulation domains, which are produced by separate wavelet transforms.
%
%Magnitude difference across domains no longer warrant difference in information content.
%
%Our research then looks into possible means that correlate with importance of domains, 
%and use those means to direct us to reallocate I/O budget.
%
%Wavelet compression is rooted in image processing where 2D images are compressed at a 
%much better efficiency than the widely used JPEG format~\cite{shapiro1993embedded,said1993image,skodras2001jpeg}.
%
In the scientific visualization community, wavelet compression is traditionally used to improve
interactivity and accelerate renderings on large data sets~\cite{kim1999efficient,ihm1999wavelet,
guthe2002interactive,treib2012turbulence}.
%
In recent years, with the rise of big data and the growing I/O gap, 
wavelets are increasingly viewed and applied as a traditional compression operator.
%
Such applications include compressing simulation checkpoint files~\cite{sasaki2015exploration},
climate model output~\cite{woodring:2011,Li:17DRBSD17},
turbulent flow simulations~\cite{Li:LDAV2015},
and more generic volumetric scientific data~\cite{villasenor1996seismic,gralka2013application,Li:CLUSTER17}.
%
%In all these applications, wavelet compression has proven to provide significant data reduction while 
%maintaining satisfactory data fidelity for further analysis.
%
%Finally, what most closely relates to this research is that wavelet compression also fits into the in situ analysis paradigm.
%
Finally, wavelet compression has been proven to fit into the in situ analysis paradigm, as it is
%That is because wavelet compression is demonstrated to be 
capable of achieving portable performance across 
modern massively parallel architectures (i.e., multi-core CPUs and GPUs)~\cite{Li:EGPGV17}
and reducing overall I/O time for simulation runs with hundreds to thousands of 
compute nodes~\cite{Li:ISAV17}.


