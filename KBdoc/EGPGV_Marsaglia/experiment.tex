
Our experiments studied a cross-product of options over three factors, for
a total of 18 tests.
%
The three factors and their options were:
\begin{itemize}[noitemsep,topsep=0pt,parsep=0pt,partopsep=0pt]
\item \textbf{Allocation Strategy}.  This factor had three options (all described in Section 3): Magnitude, Entropy, and Standard.
\item \textbf{I/O Budget}.  This factor also had three options: data reductions of 32:1, 64:1, and 128:1.
\item \textbf{Simulation Code}.  This factor had two options:
\begin{itemize}[noitemsep,topsep=0pt,parsep=0pt,partopsep=0pt]
\item Lulesh \cite{lulesh} is a hydrodynamics simulation that models the Sedov blast problem. 
%
This simulation had size $1000^3$ and ran for 10,200 cycles with 512 ranks on 32 nodes. 
%
The experiments for Lulesh were performed on Cori.
\item CloverLeaf3D \cite{clover} is a simulation that solves the compressible Euler equations. 
%
This simulation had size $1000^3$ and ran for 500 cycles with 64 ranks on 32 nodes. 
%
The experiments for CloverLeaf3D were performed on Cheyenne. 
\end{itemize}
\end{itemize}

This research was implemented within the Ascent framework~\cite{Larsen} and the experiments were run on NERSC's Cori supercomputer, as well as NCAR's Cheyenne supercomputer.
%
Ascent provides a lightweight in situ infrastructure that includes a vtk-m wavelet implementation~\cite{vtkm}. 
%

%To determine the merit of our reallocation strategies compared to the standard, fixed wavelet compression, our strategies were run in situ on two different simulations present within Ascent and with varying I/O budgets. 
%%  
%\subsection{Experiment Parameters}
%Below are the factors that we varied for each of our experiments. 
%

%\subsubsection{Simulations}
%\begin{itemize}[noitemsep,topsep=0pt,parsep=0pt,partopsep=0pt]
%\item Lulesh \cite{lulesh} is a hydrodynamics simulation that models the Sedov blast problem. 
%
%This simulation had size $1000^3$ and ran for 10,200 cycles with 512 ranks on 32 nodes. 
%%
%The experiments for Lulesh were performed on Cori.
%\item CloverLeaf3D \cite{clover} is a simulation that solves the compressible Euler equations. 
%%
%This simulation had size $1000^3$ and ran for 500 cycles with 64 ranks on 32 nodes. 
%%
%The experiments for CloverLeaf3D were performed on Cheyenne. 
%\end{itemize}

%\subsubsection{Allocation Strategies}
%\begin{itemize}[noitemsep,topsep=0pt,parsep=0pt,partopsep=0pt]
%\item Magnitude 
%\item Entropy
%\item Standard
%\end{itemize}

%\subsubsection{I/O Budget}
%\begin{itemize}[noitemsep,topsep=0pt,parsep=0pt,partopsep=0pt]
%\item 32:1
%\item 64:1
%\item 128:1
%\end{itemize}

%\subsection{Measurements}
The effects of data reduction were measured by considering the Normalized Root Mean Square Error (NRMSE) for the compressed output compared to the original data. 
We also measured the overhead to do reallocation, i.e., the amount of time spent coordinating between the nodes.
%\begin{itemize}[noitemsep,topsep=0pt,parsep=0pt,partopsep=0pt]
%\item Maximum and Average NRMSE
%\end{itemize}

 




 
